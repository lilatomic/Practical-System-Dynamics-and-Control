\documentclass{../templates/topic}

\begin{document}

\chapter{Canonical Form of 2nd Order Equations}

\begin{section}{Definition}
	The Canonical Form is the form where the parameters have some relation to thebehaviour of the system which society has deemed useful.
	
	For a system defined as
	\begin{equation}
		\frac{K}{J*s^2+B*s+K}=\frac{\frac{K}{J}}{s^2+\frac{B}{J}s+\frac{K}{J}}
	\end{equation}
	
	the cannonical form is given as
	\begin{equation}
		\frac{\omega_n^2}{s^2+2\zeta\omega_n s + \omega_n^2}
	\end{equation}
	which leads naturally to the following mapping
	\begin{align}
		\omega_n &= \sqrt{\frac{K}{J}}\\
		\zeta &= \frac{B}{2\sqrt{JK}}
	\end{align}
	
\end{section}

\begin{section}{Element Definitions}
	\definition{$\omega_n$} The natural frequency
	
	\definition{$\omega_d$} $\omega_d=\omega_n\sqrt{1-\zeta^2}$
	
	\definition{$\zeta$} The damping coefficient. Indicates the strength of the damping of the system.
	\begin{equation*}
		\begin{cases}
			\zeta \in (0,1) & subcritically damped \\
			\zeta = 1 & critically damped \\
			\zeta > 1 & supercritically damped
		\end{cases}
	\end{equation*}
	
	\definition{$\sigma$} $\zeta\omega_n$
\end{section}

\begin{section}{Application}
	% TODO: Canonical_Form_2nd_Order::Application_zeta
	% TODO: Canonical_Form_2nd_Order::Application_omegan
	% TODO: Canonical_Form_2nd_Order::Application_omegad
\end{section}

\end{document}