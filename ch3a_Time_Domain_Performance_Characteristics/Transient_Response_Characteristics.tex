\documentclass{../templates/topic}

\begin{document}
\graphicspath{{assets/}{ch3a_Time_Domain_Performance_Characteristics/assets/}}

\chapter{Transient Response Characteristics}

\begin{section}{Definitions}
	
	All of these refer in general to the Unit Step Response.
	
	\definition{rise time $\approx \frac{1.8}{\omega_n}$}  time from a percentage from the initial values to the 1st time encountering a percentage from the final value. Most commonly, it is from 0.1 of initial to 0.9 of final.
	
	\definition{peak time $\approx \frac{\pi}{\omega_d}$}  time to hit the highest value. More generally, the time until the first peak.
	
	\definition{settling time $\approx \frac{4}{\zeta\omega_n} (for 0.02)$}  time before the system has converged to within a specified percentage of the final value.
	
	\definition{overshoot $ \approx \exp[\frac{-\pi\zeta}{\sqrt{1-\zeta^2}}] $} ratio of the height of the 1st peak above the final value to the final value. Note that for overdamped systems, which begin with movement away from the final value, the overshoot can be considered the ratio of the height of that trough.
	
	\begin{figure}[H]
		\includegraphics[width=\textwidth]{characteristics_on_graph.png}
		\caption{Time Response Characteristics on a sample function}
	\end{figure}
	
\end{section}

\begin{section}{2nd Order Underdamped Systems}
	
	\begin{subsection}{Designing for Characteristics}
	\subsection{Overshoot}
		\begin{equation*}
			\zeta = \sqrt{\frac{(\ln{M_p})^2}{\pi^2+(\ln{M_p})^2}}
		\end{equation*}
		
	\end{subsection}
	
\end{section}

\end{document}