\documentclass{../templates/topic}

\begin{document}
\chapter{Electrical Systems}

\begin{section}{Impedance}
	\definition{Impedance} $Z(s)=\frac{V(s)}{I(s)}$
	
	\subsection{Network Impedance}
	Impedances function like resistances:
	\subsubsection*{Series}
	Series impedances are additive
	\begin{equation}
		Z_{eq} = \sum_{i}^{n}{R_i}
	\end{equation}
	\subsubsection*{Parallel}
	Parallel impedances are reciprocally additive
	\begin{equation}
		Z_{eq} = (\sum_{i}^{n}{\frac{1}{R_i}})^{-1}
	\end{equation}
\end{section}

\begin{section}{Elements}
	\begin{itemize}
		\item Resistor : $Z=R$
		\item Capacitor : $Z=\frac{1}{j\omega C}$
		\item Inductor : $Z=j\omega L$
		\item Voltage Generator : Supplies constant potential difference regardless of flux $Z=0$
		\item Current Generator : Supplies constant flux regardless of potential difference $Z=\infty$
		\item Transformer : $e_2 = ne_1$; $i_2 = \frac{1}{n}i_1$
		\item Gyrator : Interchange potential and flux. It is most useful conceptually, as it allows different models to be bridged. 
	\end{itemize}
\end{section}


\begin{section}{Analysis}
	There are two main techniques to use:
	\subsection{Voltage Law}
		The potential around any loop to the same point is 0
	\subsection{Current Law}
		The sum of current into any node is 0
\end{section}

\begin{section}{The Universal Analogy}
	Electrical systems will often be used as an intermediate model when analysing other systems. These systems have the Voltage and Current laws easily formulated, making analysis more straightforward.
	
	\subsection{Impedance Analog}
		The Impedance Analog defines
		\begin{itemize}
			\item Potential represents Force
			\item Flux represents Velocity
		\end{itemize}
	
	\subsection{Mobility Analog}
		The Mobility Analog defines
		\begin{itemize}
			\item Potential represents Velocity
			\item Flux represents Force
		\end{itemize}
	\subsection{Equivalence of Analogs}
		
\end{section}

\end{document}