\documentclass{../templates/topic}

\begin{document}
\chapter{Pneumatic Systems}

Pneumatic and Acoustic systems are basically the same from a modelling point of view. The only difference is the frequency of the responses we are concerned with. With pneumatic systems, this is typically low frequency; with acoustic systems, this is typically high frequency.

\section{Pneumatic Systems}

\subsection{Electrical Analogy}
Standard Items
\begin{itemize}
	\item Capacitive : Chamber of volume C
	\item Inductive :
	\item Resistive : Valve of resistance R
\end{itemize}
Additional Items
\begin{itemize}
	\item Nozzle-Flapper : Pressure to Force or Displacement (Gyrator or Transformer; arbitrary constant K)
	\item Bellows : Pressure to Force or Displacement (Gyrator or Transformer; constant is Area of bellows)
\end{itemize}

\section{Acoustic Systems}

\subsection{Electrical Analogy}
Standard Items
\begin{itemize}
	\item Capacitive : Chamber with capacity C (to ground)
	\item Inductive : pipe with inductance M
	\item Resistive : porous plug with resistance R
	\item Voltage Source : pressure generator
	\item Current Source : Volume flow generator
\end{itemize}



\end{document}