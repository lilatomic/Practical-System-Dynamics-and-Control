\documentclass{../templates/topic}

\title{Solution of State Space Problems}
\author{Daniel "lilatomic" Goldman}

\begin{document}
\maketitle

\begin{section}{Zero Input Solution}
	
	\subsection{Definition}
	
		For a system defined as
		\begin{align}
			\dot{q}&=Aq+Bu \\
			y&= Cq+Du
		\end{align}
		the Zero Input Problem is given as
		\begin{align}
			\dot{q}&=Aq \\
			y&= Cq
		\end{align}
		on account of the input being zero and all that.
		
	\subsection{Solution}
		We begin by taking the $\laplace$.
		We then solve for $Q(s)$. The shortmode gives
		\begin{equation}
			Q=\Phi\cdot q(0^-)
		\end{equation}
		The inverse laplace transform gives
		\begin{equation}
			q=\phi\cdot q(0^-)
		\end{equation}
		
		Which can be substituted back into the equation for $y$ to give
		\begin{equation}
			y(t)=C\cdot q(t)=C\phi\cdot q(0^-)
		\end{equation}
	
\end{section}

\begin{section}{Zero State Solution}
	
\end{section}

\begin{section}{General Solution}
	
\end{section}

\begin{section}{References}
	http://lpsa.swarthmore.edu/Transient/TransMethSS.html
\end{section}

\end{document}