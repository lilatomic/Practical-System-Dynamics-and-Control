\documentclass{../templates/topic}

\begin{document}

\chapter{Partial Fraction Decomposition}


\begin{section}{Definition}
	% TODO: add section Partial Fraction Decomposition::Definition
\end{section}

\begin{section}{Special Cases}
	\begin{subsection}{Complex Conjugate}
		With a second-order or higher denominator, it is possible that the denominator has complex factors.
		One solution is to simply accept that the roots will be complex and use complex conjugates:
		\begin{equation*}
			\frac{U}{s^2+2s+5}=\frac{A}{s+1-2\hat{j}}+\frac{B}{s+1+2\hat{j}}
		\end{equation*}
		Alternatively, you can leave it uncomplexified, and simply as a second order term.
		\begin{equation*}
			\frac{U}{s^2+1}=\frac{C_1s+C_0}{s^2+1}
		\end{equation*}
		
		Doing this makes it easy to rearange it into a form amenable for $\sin$ or $\cos$.
		\begin{equation*}
			\frac{C_1s+C_0}{s^2+2s+5}=\frac{C_1s}{(s+1)^2+2^2}+\frac{C_0}{(s+1)^2+2^2}
		\end{equation*}
		Or you can just use the table of common $\ilaplace$ to not have to do the rearanging.
	\end{subsection}
	
	\begin{subsection}{Repeated Factor}
		% TODO: Add subsection: Partial Fraction Decomposition::Repeated Factor
	\end{subsection}
	\begin{subsection}{Improper Fraction}
		A fraction is improper when the degree of the numerator is not less than the degree of the denominator. The degree of mismatch is the difference between the former and the latter.
		
		In the case of an improper fraction, a polynomial of the degree of mismatch must be added to the PDF side.
		
		\begin{align*}
			\frac{(s+1)^2}{(s-2)^2} &= \frac{A}{(s-2)^2}+\frac{B}{(s-2)}+\textcolor{blue}{C} \\
			&= \frac{9}{(s-2)^2}+\frac{6}{(s-2)}+\textcolor{blue}{1}
		\end{align*}
		
		% TODO: Add higher order example for Partial Fraction Decomposition::Improper Fraction
		
	\end{subsection}
\end{section}
	

\end{document}