\documentclass{../templates/topic}

\begin{document}

\chapter{Partial Fraction Decomposition}


\begin{section}{Definition}
	% TODO: add section Partial Fraction Decomposition::Definition
\end{section}

\begin{section}{Special Cases}
	\begin{subsection}{Repeated Factor}
		% TODO: Add subsection: Partial Fraction Decomposition::Repeated Factor
	\end{subsection}
	\begin{subsection}{Improper Fraction}
		A fraction is improper when the degree of the numerator is not less than the degree of the denominator. The degree of mismatch is the difference between the former and the latter.
		
		In the case of an improper fraction, a polynomial of the degree of mismatch must be added to the PDF side.
		
		\begin{align*}
			\frac{(s+1)^2}{(s-2)^2} &= \frac{A}{(s-2)^2}+\frac{B}{(s-2)}+\textcolor{blue}{C} \\
			&= \frac{9}{(s-2)^2}+\frac{6}{(s-2)}+\textcolor{blue}{1}
		\end{align*}
		
		% TODO: Add higher order example for Partial Fraction Decomposition::Improper Fraction
		
	\end{subsection}
\end{section}
	

\end{document}